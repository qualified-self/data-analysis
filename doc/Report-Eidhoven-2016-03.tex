\documentclass[a4paper,10pt]{article}
\usepackage[T1]{fontenc}
\usepackage[utf8]{inputenc}

\usepackage{textcomp}
\usepackage[english]{babel}

\usepackage[%
        backend=biber,
        style=chicago-authordate,
        natbib=true,
        backref=false,
        backrefstyle=all+,
        hyperref=true,
    ]{biblatex}
%\usepackage{natbib}

\usepackage{setspace}
\setstretch{1.5}
%\setstretch{2}

\usepackage{subfig}
\usepackage{float}

\usepackage{pstricks}

\usepackage{tabularx}
\usepackage{makecell}

\usepackage{geometry}
\usepackage{auto-pst-pdf}
%\usepackage{graphicx}
%\usepackage{subfigure}
\usepackage[colorlinks=true,linkcolor=black,citecolor=black,urlcolor=black,filecolor=black]{hyperref}
\usepackage{url}

\usepackage{algorithm}
\usepackage{algpseudocode}

\usepackage{amsmath}

\usepackage{lscape}

\usepackage{listings}
\usepackage{color}

\definecolor{dkgreen}{rgb}{0,0.6,0}
\definecolor{gray}{rgb}{0.5,0.5,0.5}
\definecolor{mauve}{rgb}{0.58,0,0.82}

\lstset{frame=tb,
  language=C++,
  aboveskip=3mm,
  belowskip=3mm,
  showstringspaces=false,
  columns=flexible,
  basicstyle={\small\ttfamily},
  numbers=none,
  numberstyle=\tiny\color{gray},
  keywordstyle=\color{blue},
  commentstyle=\color{dkgreen},
  stringstyle=\color{mauve},
  breaklines=true,
  breakatwhitespace=true,
  tabsize=3
}

\newenvironment{blockquote}{\begin{quotation}\singlespacing}{\end{quotation}}

%\bibliographystyle{apalike} 
\bibliography{qself}

%opening
\title{Technical Report (internal)}
\author{Sofian Audry}

\begin{document}

\maketitle

\begin{abstract}
This tech report describes a series of experiments that were conducted in Eidhoven from March 30\textsuperscript{th} to April 1\textsuperscript{st}, 2016. The experiments involved the measurement of three body signals (ECG, GRS and respiration) in groups of subjects. The objective of the study was to examine the synchrony of body signals in a group of people when subjected to different sets of conditions.
\end{abstract}

\section{Measures}

We apply the cluster-phase synchrony measurement method proposed in \citep{Richardson2012-Measuring} on groups of between 4 and 6 human subjects.

\section{Experiments}

\subsection{2016-03-31 Exp. 1-2-3}

The three experiments we performed on 2016-03-31 had the following basic frame:

\begin{table}[h]
\caption{Basic frame of 2016-03-31 experiments}
\begin{tabularx}{\textwidth}{|c|c|X|} \hline
Phase & Duration (s) & Description\\\hline\hline 
A        & 300              & Nothing happens (baseline)\\\hline
B        & 300              & Instruction for synced breathing and silence\\\hline
C        & 600              & Stimuli (varies, see below)\\\hline
D        & 300              & Nothing happens (relaxation phase)\\\hline
\end{tabularx}
\end{table}

Phases A, B, and D were all the same: however phase C was changed to measure the effects of different kinds of stimuli:

\begin{table}[h]
\caption{Phase C changes in 2016-03-31 experiments}
\begin{tabularx}{\textwidth}{|c|X|} \hline
Exp. \# & Phase C description\\\hline\hline 
1        & Entraining sound of Sea Waves\\\hline
2        & Haptic repetitive stimuli, increasing amplitude over 10 min then decreasing over 5 min\\\hline
3        & One heartbeat mapped to haptics\\\hline
\end{tabularx}
\end{table}

The experiments were done over 2 groups of people (group A and group B).

\section{Results}

Results are presented below. In general, it might appear that cluster-phase group synchrony was affected by the different phases, however in ways that are not conclusive. Moreover, some of the data that was gathered seems suspicious.

%\begin{landscape}
 
\begin{figure}[hbtp]
\centering

\subfloat[Group A Exp. \#1 (ECG)]{\includegraphics[keepaspectratio,width=0.3\textwidth,height=0.3\textheight]{images/data_exp2016-03-31--01-03_ECG_cluster_245760-1658880.png}}
\qquad
\subfloat[Group A Exp. \#1 (GSR)]{\includegraphics[keepaspectratio,width=0.3\textwidth,height=0.3\textheight]{images/data_exp2016-03-31--01-03_GSR_cluster_245760-1658880.png}}
\qquad
\subfloat[Group A Exp. \#1 (RESP)]{\includegraphics[keepaspectratio,width=0.3\textwidth,height=0.3\textheight]{images/data_exp2016-03-31--01-03_RESP_cluster_245760-1658880.png}}

\subfloat[Group A Exp. \#2 (ECG)]{\includegraphics[keepaspectratio,width=0.3\textwidth,height=0.3\textheight]{images/data_exp2016-03-31--01-03_ECG_cluster_1658880-3072000.png}}
\qquad
\subfloat[Group A Exp. \#2 (GSR)]{\includegraphics[keepaspectratio,width=0.3\textwidth,height=0.3\textheight]{images/data_exp2016-03-31--01-03_GSR_cluster_1658880-3072000.png}}
\qquad
\subfloat[Group A Exp. \#2 (RESP)]{\includegraphics[keepaspectratio,width=0.3\textwidth,height=0.3\textheight]{images/data_exp2016-03-31--01-03_RESP_cluster_1658880-3072000.png}}

\subfloat[Group A Exp. \#3 (ECG)]{\includegraphics[keepaspectratio,width=0.3\textwidth,height=0.3\textheight]{images/data_exp2016-03-31--01-03_ECG_cluster_3072000-4608000.png}}
\qquad
\subfloat[Group A Exp. \#3 (GSR)]{\includegraphics[keepaspectratio,width=0.3\textwidth,height=0.3\textheight]{images/data_exp2016-03-31--01-03_GSR_cluster_3072000-4608000.png}}
\qquad
\subfloat[Group A Exp. \#3 (RESP)]{\includegraphics[keepaspectratio,width=0.3\textwidth,height=0.3\textheight]{images/data_exp2016-03-31--01-03_RESP_cluster_3072000-4608000.png}}

\caption{Cluster-phase analysis of the first set of experiments conducted on March 31\textsuperscript{st}, 2016 for electrocardiogram (ECG), galvanic skin response (GSR) and respiration (RESP). Top graph displays raw data; middle graph shows the individual cluster-phases; bottom graph contains the group (average) cluster phase. The four phases (A, B, C, D) are marked using dotted vertical lines.}
\end{figure}
%\end{landscape}

\begin{figure}[hbtp]
\centering

\subfloat[Group B Exp. \#1 (ECG)]{\includegraphics[keepaspectratio,width=0.3\textwidth,height=0.3\textheight]{images/data_exp2016-03-31--04_ECG_cluster_0-1798143.png}}
\qquad
\subfloat[Group B Exp. \#1 (GSR)]{\includegraphics[keepaspectratio,width=0.3\textwidth,height=0.3\textheight]{images/data_exp2016-03-31--04_GSR_cluster_0-1798143.png}}
\qquad
\subfloat[Group B Exp. \#1 (RESP)]{\includegraphics[keepaspectratio,width=0.3\textwidth,height=0.3\textheight]{images/data_exp2016-03-31--04_RESP_cluster_0-1798143.png}}

\iffalse
\subfloat[Group B -- Experiment \#2 (ECG)]{\includegraphics[keepaspectratio,width=0.3\textwidth,height=0.3\textheight]{images/data_exp2016-03-31--05_ECG_cluster_0-1508351.png}}
\qquad
\subfloat[Group B -- Experiment \#2 (GSR)]{\includegraphics[keepaspectratio,width=0.3\textwidth,height=0.3\textheight]{images/data_exp2016-03-31--05_GSR_cluster_0-1508351.png}}
\qquad
\subfloat[Group B -- Experiment \#2 (RESP)]{\includegraphics[keepaspectratio,width=0.3\textwidth,height=0.3\textheight]{images/data_exp2016-03-31--05_RESP_cluster_0-1508351.png}}

\subfloat[Group B -- Experiment \#3 (ECG)]{\includegraphics[keepaspectratio,width=0.3\textwidth,height=0.3\textheight]{images/data_exp2016-03-31--01-03_ECG_cluster_3072000-4608000.png}}
\qquad
\subfloat[Group B -- Experiment \#3 (GSR)]{\includegraphics[keepaspectratio,width=0.3\textwidth,height=0.3\textheight]{images/data_exp2016-03-31--01-03_GSR_cluster_3072000-4608000.png}}
\qquad
\subfloat[Group B -- Experiment \#3 (RESP)]{\includegraphics[keepaspectratio,width=0.3\textwidth,height=0.3\textheight]{images/data_exp2016-03-31--01-03_RESP_cluster_3072000-4608000.png}}

\fi

\caption{Cluster-phase analysis of the second set of experiments conducted on March 31\textsuperscript{st}, 2016 for electrocardiogram (ECG), galvanic skin response (GSR) and respiration (RESP). Top graph displays raw data; middle graph shows the individual cluster-phases; bottom graph contains the group (average) cluster phase. The four phases (A, B, C, D) are marked using dotted vertical lines.}
%\caption{Cluster-phase analysis of the second set of experiments conducted on March 31\textsuperscript{st}, 2016 for electrocardiogram (ECG), galvanic skin response (GSR) and respiration (RESP). Top graph displays raw data; middle graph shows the individual cluster-phases; bottom graph contains the group (average) cluster phase.}
\end{figure}

\section{Conclusion}

I do not think we can really conclude anything from these measurements. It is possible that we do not have the right measurements, so it might be one thing to look at. More experiments would also be required under more controlled conditions since some of the data is unreliabled due to different kinds of perturbations (talking, failure of server, unplugged sensors, etc).

\printbibliography

\end{document}
